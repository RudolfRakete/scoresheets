\documentclass{standalone}
\usepackage{multirow}
\usepackage{booktabs}
\usepackage[table]{xcolor}

\newcommand\myhline{\cmidrule(r){1-2}\cmidrule(lr){3-6}\cmidrule(lr){7-12}\cmidrule(l){13-18}}
%\newcommand\myhline{\cmidrule(r){1-2}\cmidrule(lr){3-6}\cmidrule(lr){7-9}\cmidrule(lr){10-12}\cmidrule(lr){13-15}\cmidrule(l){16-18}}
\def\arraystretch{1.2}
\begin{document}
   \begin{tabular}{l}
   TITLE\\
   \rowcolors{2}{gray!10}{white}
      \begin{tabular}{ ll r rrr rrr rrr rrr rrr}
\toprule
         && \multicolumn{4}{c}{Punkte} & \multicolumn{6}{c}{Sätze} & \multicolumn{6}{c}{Spiele}\\%\myhline
         Name & Nummer(n) & anwesend & gespielt & \%\,insg. & \%\,anw. & beteiligt & \%\,insg. & \%\,anw. & gestartet & \%\,insg. & \%\,anw. & beteiligt & \%\,insg. & \%\,anw. & gestartet & \%\,insg. & \%\,anw. \\\myhline
      TABLE
      \bottomrule
      \end{tabular}
   \end{tabular}
\end{document}
